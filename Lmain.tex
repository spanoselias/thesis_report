%%%%%%%%%%%%%%%%%%%%%%%%
%% Sample use of the infthesis class to prepare a thesis. This can be used as 
%% a template to produce your own thesis.
%%
%% The title, abstract and so on are taken from Martin Reddy's csthesis class
%% documentation.
%%
%% MEF, October 2002
%%%%%%%%%%%%%%%%%%%%%%%%

%%%%
%% Load the class. Put any options that you want here (see the documentation
%% for the list of options). The following are samples for each type of
%% thesis:
%%
%% Note: you can also specify any of the following options:
%%  logo: put a University of Edinburgh logo onto the title page
%%  frontabs: put the abstract onto the title page
%%  deptreport: produce a title page that fits into a Computer Science
%%      departmental cover [not sure if this actually works]
%%  singlespacing, fullspacing, doublespacing: choose line spacing
%%  oneside, twoside: specify a one-sided or two-sided thesis
%%  10pt, 11pt, 12pt: choose a font size
%%  centrechapter, leftchapter, rightchapter: alignment of chapter headings
%%  sansheadings, normalheadings: headings and captions in sans-serif
%%      (default) or in the same font as the rest of the thesis
%%  [no]listsintoc: put list of figures/tables in table of contents (default:
%%      not)
%%  romanprepages, plainprepages: number the preliminary pages with Roman
%%      numerals (default) or consecutively with the rest of the thesis
%%  parskip: don't indent paragraphs, put a blank line between instead
%%  abbrevs: define a list of useful abbreviations (see documentation)
%%  draft: produce a single-spaced, double-sided thesis with narrow margins
%%
%% For a PhD thesis -- you must also specify a research institute:
%%\documentclass[phd,ilcc,twoside]{infthesis}

%% For an MPhil thesis -- also needs an institute
% \documentclass[mphil,ianc]{infthesis}

%% MSc by Research, which also needs an institute
% \documentclass[mscres,irr]{infthesis}

%% Taught MSc -- specify a particular degree instead. If none is specified,
%% "MSc in Informatics" is used.
% \documentclass[msc,cogsci]{infthesis}
\documentclass[msc, logo]{infthesis}  % for the MSc in Informatics

%% Master of Informatics (5 year degree)
% \documentclass[minf]{infthesis}

%% Undergraduate project -- specify the degree course and project type
%% separately
% \documentclass[bsc]{infthesis}
% \course{Artificial Intelligence and Psychology}
% \project{Fourth Year Project Report}

%% Put any \usepackage commands you want to use right here; the following is 
%% an example:
\usepackage{natbib}
\usepackage{graphicx}
\usepackage{amsmath}
\usepackage{array}
\usepackage{multirow}
\usepackage[table,xcdraw]{xcolor}

%%Package for bullet
\usepackage{enumitem }
%%Package for tick icon in the experiments chapter
\usepackage{pifont}

%%package
\usepackage{listings}

\usepackage{xcolor}
\usepackage{mdframed}

%% This library is used to make 
%% Clickable table of content 
\usepackage{hyperref}
\hypersetup{
    colorlinks,
    citecolor=black,
    filecolor=black,
    linkcolor=black,
    urlcolor=black
}
 
%% Information about the title, etc.
\title{ Testing SQL-compliance of current DBMSs}
\author{Elias Spanos}

%% If the year of submission is not the current year, uncomment this line and 
%% specify it here:
% \submityear{1785}

%% Optionally, specify the graduation month and year:
% \graduationdate{February 1786}

%% Specify the abstract here.
\abstract{%
Database Management Systems (DBMSs) are widely used in various fields such as the financial sector, the academia and other online services and are very crucial in the day to day management of data. The correct and efficient operation of such systems is an important factor when choosing the right DBMS software. In the past few decades, the amount of data has increased exponentially, causing a rapid increase in the demand for software that can store, organise and manipulate such data. With this requirements in mind, databases and DBMSs such as Oracle DB, Microsoft SQL Server, MySQL and PostgreSQL were created, each implementing its own set of rules. However, since these DBMSs have been implemented with a significant amount of differences, a set of standards was set by the American National Standards Institute (ANSI) in 1986 in order to keep all these implementations to a same standard and ensure compatibility for the SQL language adopted in all implementations. Even though, the SQL Standard has been introduced and adopted by most of the implementations, there are some differences that still exist, probably due to different interpretations of or noncompliance to the aforementioned Standards, and also due to other issues regarding performance of their systems.Since there do exist some differences between each implementation, migrations and changes from one DBMS to another might lead to some bottlenecks such as incompatibility or semantic issues. Therefore, there is a need for alleviating such a problem by exposing such issues between DBMSs, offering proposed solutions that have minimal negative effects to the performance of a system and also for evaluating their conformance to the common Standard. This project will investigate five different implementations and evaluate their conformance to the SQL Standard.  A crucial question that will be investigated by conducting this project is whether DBMSs have been implemented the SQL standards in the same way. The SQL language should pledge that identical SQL code should always return identical answers when it is evaluated on the same database independently of which DBMS is running on. 

The aim of this project is the implementation of a random SQL query generator and a comparison tool for investigating and highlighting the differences that may exist among current DBMSs. Further, we aim to provide a detailed explanation in regards to SQL standards of potential differences and explain how they might affect the transition between current DBMSs. 
 
}

%% Now we start with the actual document.
\begin{document}

%% First, the preliminary pages
\begin{preliminary}

%% This creates the title page
\maketitle

%% Acknowledgements
\begin{acknowledgements}
 
I would like to thank my supervisors Paolo Guagliardo and Leonid Libkin who were always willing to advise me and help me in order to overcome any difficulty. In addition, I want to thank my family who is always by my side 

\end{acknowledgements}

%% Next we need to have the declaration.
\standarddeclaration

%% Finally, a dedication (this is optional -- uncomment the following line if
%% you want one).
% \dedication{To my mummy.}

%% Create the table of contents
\tableofcontents

%% If you want a list of figures or tables, uncomment the appropriate line(s)
%%\listoffigures
%%\listoftables

\end{preliminary}

%%%%%%%%
%% Include your chapter files here. See the sample chapter file for the basic
%% format.

\include{appendix1}
 
\chapter{Introduction}
 \section{Motivation}


DBMSs are widely used in many fields and find application in many companies as they provide relatively easy way of performing various common operations on data, such as insertion, deletion and update, and at the same time they hide the internal complexity of such  system. More precisely, DBMS is a software that is designed to allow the creation, querying and update of databases [2,3]. The main role of a DBMS is to store and manipulate data efficiently and consistently. Figure 1.1 shows in a high level the structure of modern DBMSs. Additionally, Structured Query language  known as SQL is a standardised programming language for managing relational databases [1]. By having such standards, it can be provided easily a common interface for all DBMSs in order to manipulate and retrieve data from any given database without worrying about the internal implementation. 
 
The aim of this project is to evaluate five systems such as MySQL, IBM DB2, Microsoft SQL Server, PostgreSQL and Oracle Database in order to identify if they interpret the SQL standards in exactly the same way. 

 \begin{figure} 
      \centering
      \includegraphics[width=\textwidth]{Images/db_architecture}
      \caption{DBS architecture}
      \label{fig:counting-methods}
    \end{figure}

As it was mentioned all current DBMSs currently support the SQL standards, meaning that there is a common language, namely SQL, that is used by all systems to access, manipulate and retrieve data. Nevertheless some aspects of the standards are not well defined which make the process of interpret and implement it a difficult task and as a results, companies implement the SQL standards in a different manner. As a consequence programs that are written in SQL are partially portable among different DBMS.  



\begin{table}[h]
\centering
\caption{My caption}
\label{my-label}
\begin{tabular}{lll}
\multicolumn{3}{l}{\textbf{Is it a bug?}}                                                                               \\
                                       & \multicolumn{1}{c}{\textbf{Y}}           & \multicolumn{1}{c}{\textbf{N}}      \\ \cline{2-3} 
\multicolumn{1}{l|}{\textbf{Bugs}}     & \multicolumn{1}{l|}{True Positive (TP)}  & \multicolumn{1}{l|}{False positive} \\ \cline{2-3} 
\multicolumn{1}{l|}{\textbf{Reported}} & \multicolumn{1}{l|}{False Negative (FN)} & \multicolumn{1}{l|}{True Negative}  \\ \cline{2-3} 
                                       &                                          &                                    
\end{tabular}
\end{table}


\begin{math}
\frac{2}{3}
\end{math}
\chapter{Background}

\section{The SQL Standard }
DBMS from its first appearance shows that it will be the dominant trend for managing data. Consequently, different implementations have emerged from various vendors and inevitably, a standardised language should be implemented in order to provide portability among current systems. If applications were implemented using only SQL commands which are defined in that standard and vendors implemented these commands in exactly the same way, then SQL code could be migrated on any DBMS without the need to be adopted.

The first appearance of the SQL language was in 1970 where IBM developed the first prototype of Relational Database Management System (RDBMS). Subsequently, the first SQL standard arose in 1986 by the American National Standard Institute (ANSI) with the name SQL-86 for bearing conformity among vendors’ implementations. Since then, different flavors of the standard have being emerged for revising previous versions or for adding new features such as SQL-87, SQL-89, ANSI/ISO SQL-92 and ANSI/ISO SQL: 1999 which has been approved also by the International Standards Organization (ISO) [11]. The SQL Standard has been continuously developed with current version being SQL:2016 or ISO/IEC 9075:2016. The ANSI SQL standard is divided into several parts and this project focuses on the SQL/Foundation part. This part contains central elements of SQL. Explanations about the findings are explained according to SQL:2016 since it is the newest version of the standard, though this part  remains the same in comparison with earlier flavors. 


\section{The SQL Language}
SQL operations are in the form of commands known as SQL statements. More precisely, SQL is composed of primarily two sublanguages, the data definition language (DDL) and the data manipulation language (DML). DDL is a part of SQL language and can be used to create, modify, delete tables and views, and usually DDL statements start with keywords ‘CREATE’, ‘DROP’ and ‘ALTER’. Also, it supports a command that gives the capability of defining new domains. Moreover, in general, tables and rows are denoted as relations and tuples. DML is also a part of the SQL language that is composed of a family of commands like any programming language and is used for the creation of a query for inserting, modifying and deleting rows in a Database. This sublanguage is consisted by ‘SELECT-FROM-WHERE’ commands as to be the fundamental for any query. In addition, the SQL Standard supports more complex rather than just simple commands for performing various tasks on data. For example, aggregation functions such as ‘Sum’, ‘Max’, ‘Min’, ‘Avg’ are used with combination with ‘Group By’, and ‘Having’ SQL statements. The purpose of having such commands is to perform a calculation on a specific column in order to return a value, for example the calculation of the average salary of a department by performing a ‘Group By’ based on all the salaries of the employees of that department. 
Hence, SQL is extremely popular as it offers two capabilities. Firstly, it can access many tuples using just one command and secondly it does not need to specify how to reach a tuple, for example by using an index or not. 


\section{Commands of SQL}
This sections intends to provide a high-level description of the basic commands of SQL as are defined in the Standard. As these commands are used to generate random SQL queries, it is important to provide a basic background of the usage and purpose of each command. 

\hfill\newline
\noindent\textbf{\underline{SQL Basic Structure}} 
\begin{mdframed}[nobreak=true, backgroundcolor=lightgray!20] 
\begin{lstlisting}[style=SQL]
SELECT [DISTINCT] columns_list
FROM tables_list
WHERE Condition1 {AND|OR} Condition 2
\end{lstlisting}
\end{mdframed}

 
The SQL basic structure represents the basic commands which are used to retrieve data from a database based on some criterias that can be specified in the WHERE clause. Each SQL query should have at least a SELECT and FROM clause.  The WHERE clause is optional. The basic SQL is executed as follow: It goes through all the rows of the tables list in the FROM Clause. Then, each row that satisfied the search criteria is selected. Then, only the specified columns of the selected rows appear in the result. In addition,  the DISTINCT keyword is optional and can used to eliminate duplicates rows in the result. Also, instead of having columns\_list in the SELECT clause, the * keyword can be used which indicates that all the columns of the tables\_list will appear in the result. The usage of the WHERE clause is to filter the results and usually SQL queries contains a WHERE clause. Conditions are used to compare expressions using comparison operators. A comparison operator is used to compare two expressions and logical connectivities, such as AND, NOT and OR are used to connect the conditions. 

\hfill\newline
\noindent\textbf{\underline{Example of basic query:}}
\begin{mdframed}[backgroundcolor=gray!20] 
\begin{verbatim}
SELECT St.Student_Name
FROM Students AS St 
WHERE St.age >= 20 AND St.age <= 24
\end{verbatim}
\end{mdframed}

The above example uses the basic SQL structure to build a query. Thus, it retrieves all the Students and based on the criteria of the ‘WHERE’ clause, it returns the name of students who are between 20 and 24 year old. 

\hfill\newline
\noindent\textbf{\underline{Example:}}
\begin{mdframed}[backgroundcolor=lightgray!20] 
\begin{lstlisting}[style=SQL]
SELECT       [DISTINCT]  Columns_list
FROM         Tables_list
WHERE       Condition1 {AND|OR} Condition2
GROUP BY Columns_list
HAVING      Condition1 {AND|OR} Condition 2
\end{lstlisting}
\end{mdframed}

SQL query with aggregation is used to perform a calculation on specific columns in order to return a value. ‘GROUP BY’ and ‘HAVING’ are used to perform an aggregation. ‘HAVING’ is an optional clause, where aggregation can still be performed using aggregate commands in the ‘SELECT’ clause. A concrete example is given subsequently. 
 

\begin{table}[h]
\centering
\caption{My caption}
\label{my-label}
\begin{tabular}{|p{2cm}|p{11.5cm}| }
\hline
\multicolumn{2}{|l|}{\textbf{Aggregate Commands}}                                                             \\ \hline
\textbf{Command}                        & \textbf{Usage}                                                      \\ \hline
{\color[HTML]{333333} \textbf{MIN()}}   & {\color[HTML]{333333} Finds the minimum value,of a column}          \\ \hline
{\color[HTML]{333333} \textbf{COUNT()}} & {\color[HTML]{333333} Counts the total number,of rows}              \\ \hline
{\color[HTML]{333333} \textbf{MAX()}}   & {\color[HTML]{333333} Finds the maximum value,of a column}          \\ \hline
{\color[HTML]{333333} \textbf{SUM()}}   & {\color[HTML]{333333} Calculates the sum of,values of a column}     \\ \hline
{\color[HTML]{333333} \textbf{AVG()}}   & {\color[HTML]{333333} Calculates the average of,values of a column} \\ \hline
\end{tabular}
\end{table}

Aggregate commands can be used both in ‘SELECT’ and ‘HAVING’ clause in combination with the ‘GROUP BY’ clause in an SQL query. The example below illustrates the proper use of aggregation commands.

\hfill\newpage
\noindent\textbf{\underline{Example:}}
\begin{mdframed}[backgroundcolor=lightgray!20] 
\begin{lstlisting}[style=SQL]
SELECT COUNT(St.student_id), St.Country
FROM Students AS St 
GROUP BY St.Students
HAVING COUNT(St.student_id)  >  3
\end{lstlisting}
\end{mdframed}
The query above makes proper uses of the aggregation commands. More precisely, the query above, list the number of students in each country where there are more than three students in a specific Country. 

\noindent\textbf{Logical  Operators:}
\begin{table}[h]
\centering
\caption{My caption}
\label{my-label}
\begin{tabular}{|p{2cm}|p{11.5cm}| }
\hline
\textbf{Operator} & \textbf{Usage}                                                     \\ \hline
EXISTS            & Returns true, if there is at least one row in the subquery         \\ \hline
Op ALL            & Returns true, if all the comparisons the operator OP return true   \\ \hline
Op ANY            & Returns true, if at least one comparison the operator returns true \\ \hline
Op IN             & Returns true, if an element exist in a given set                   \\ \hline
\end{tabular}
\end{table}

\hfill\newpage
\noindent\textbf{\underline{Example:}}
\begin{mdframed}[backgroundcolor=lightgray!20] 
\begin{lstlisting}[style=SQL]
SELECT  *
FROM Students AS St 
WHERE St.Country IN (‘UK’, ‘Netherland’)
\end{lstlisting}
\end{mdframed}
The query above  retrieves all the students who come from UK and Netherland.


\noindent\textbf{ SET Commands}
\begin{tabular}{c:cc} 
   \textbf{Command} & \textbf{Usage}  \\ \cdashline{1-2}
   UNION [ALL] & Returns the combination of the results of two SQL 	  queries \\ 
   INTERSECT [ALL] & Return the combination of the results of two 	 SQL queries for rows that appear in both results   \\ 
   EXCEPT [ALL] & Return each row that appear to the first query 	but does not appear to the second query  \\
   \hline
\end{tabular}

By default SET commands remove duplicates in the SQL result. Nevertheless, we can have duplicates in the result by adding the optional ‘ALL’ keyword for any of the commands above.  

\noindent\textbf{\underline{Example:}}
\begin{mdframed}[backgroundcolor=lightgray!20] 
\begin{lstlisting}[style=SQL]
SELECT Country FROM Students 
UNION
SELECT Country  FROM  Professor
\end{lstlisting}
\end{mdframed}
The above query retrieves the Countries where there both Students and professors. 

\begin{table}[h]
\centering
\caption{My caption}
\label{my-label}
\begin{tabular}{|p{2cm}|p{11.5cm}| }
\hline
\textbf{Operator} & \textbf{Usage}                                         \\ \hline
TRIM()            & Returns the string without leading/trailing characters \\ \hline
SUBSTRING()       & Retrieves a subset of the initial string               \\ \hline
CONCAT()          & Concatenate two or more strings                        \\ \hline
REPLACE()         & Replaces a subset of a string with another string      \\ \hline
LIKE              & Returns true, if an attribute matches with a pattern   \\ \hline
\end{tabular}
\end{table}


\noindent\textbf{\underline{Example:}}
\begin{mdframed}[backgroundcolor=lightgray!20] 
\begin{lstlisting}[style=SQL]
SELECT SUBSTRING (“SQLSTANDARD”, 1, 3 ) AS SQLExtraction 
\end{lstlisting}
\end{mdframed}

The SQL query above extract from a string which is given as parameter to the SUBSTRING function the first three characters starting from the position 1. Thus, the results is: “SQL”

\begin{table}[h]
\centering
\caption{Data Type}
\begin{tabular}{|p{2cm}|p{11.5cm}| }
\hline
\textbf{Types} & \textbf{Description}                                                                     \\ \hline
SMALLINT       & Can store a number from a range between -32768 and 32767                                 \\ \hline
INT            & Can store a number from a range between -2147483648 and 2147483647                       \\ \hline
BIGINT         & Can store a number from a range between -9223372036854775808 and 9223372036854775807     \\ \hline
VARCHAR        & Takes as input the length of a variable string which can contains up to 255 characters   \\ \hline
CHAR           & Takes as input the length of a fixed size string which can contains up to 255 characters \\ \hline
\end{tabular}
\end{table}


\textbf{Operator Precedence}

Operator precedence is an important concept for understanding how an SQL query is evaluated and it is also important for determining if major DBMSs follow the same operator precedence. There are cases where the expressions in the WHERE clause are quite complicated and operator precedence define in which order the expression should be evaluated. The sequence of operators is provided as follow, starting with higher priority and ending with lower priority: 

\begin{itemize}
\item  () 
\item  +, - , ~
\item  *, /, % 
\item  = , > , < , >= , <= , <>
\item   
\item  NOT ,  AND
\item  ALL , ANY , IN , LIKE
\item  = - variable assignment. 
\end{itemize}

Operators that have the same priority are evaluated from left to the right. In addition, parenthesis abrogate the priority of the rest operators and as a result expressions that enclosed by a parenthesis are evaluated first.    
 
 
\subsection{Missing values} 

This section aims to provide a basic background regarding NULLs and introduce the problems that can be arised from having NULLs in a DBMS. In the section of experiments, it is illustrated that many problems can be appeared by using NULLs. SQL uses NULL marker for missing or unknown values in a database and for that reason NULL is a reserved word. It worthy mentioning that NULL should not be confused with a value of zero or an empty string. Nevertheless, Oracle treats the empty string as NULL [12].  An important consideration is that it cannot be tested if a value of a field is NULL using usual comparison operators such as $ <>, = $ and $<$ but instead IS NOT NULL and IS NULL commands are used. In general the existence of NULL is the fundamental source of issues and incompatibilities among current DBMS. For evaluating each comparison with the existence of NULLS a three-valued logic (3VL) is proposed which is an extension of common boolean logic. In boolean logic, there are two values that an expression can be evaluated, namely, TRUE AND FALSE, where the negation evaluate to the opposite values. On the contrary with 3VL, in 3VL there is an addition value called unknown and the opposite of it remains the same. In addition, all comparisons involving NULL should be resulted to be unknown according to SQL Standard. Below it is illustrated a truth table for the different comparisons with the suitable outcomes.    


\begin{table}[h]
\centering
\caption{Three-Valued logic truth table}
\label{my-label}
\begin{tabular}{|l|l|l|l|l| }
\hline
\textbf{Y} & \textbf{Z} & \textbf{Y OR Z} & \textbf{Y and Z} & \textbf{NOT Y} \\ \hline
True       & True       & True            & True             & False          \\ \hline
True       & False      & True            & False            & False          \\ \hline
True       & Unknown    & True            & Unknown          & False          \\ \hline
False      & True       & True            & False            & True           \\ \hline
False      & False      & False           & False            & True           \\ \hline
False      & Unknown    & Unknown         & False            & True           \\ \hline
Unknown    & True       & True            & Unknown          & Unknown        \\ \hline
Unknown    & False      & Unknown         & False            & Unknown        \\ \hline
Unknown    & Unknown    & Unknown         & Unknown          & Unknown        \\ \hline
\end{tabular}
\end{table}

In an SQL query each tuple which is evaluated to true is returned in the result, and tuples that are evaluated to false or unknown are not returned. Comparisons that involving NULLs are considered as False in the WHERE Clause. Hence, If we take into consideration that only Oracle’s db treats empty string as NULL then it can be realized that many problem can be arised. 

\textbf{Examples}

\hfill\newline
\begin{mdframed}[nobreak=true, backgroundcolor=lightgray!20] 
\begin{lstlisting}[style=SQL]
NULL  =  10  is evaluates to Unknown 
\end{lstlisting}
\end{mdframed}

\hfill\newline
\begin{mdframed}[nobreak=true, backgroundcolor=lightgray!20] 
\begin{lstlisting}[style=SQL]
NULL = NULL  is evaluates to Unknown 
\end{lstlisting}
\end{mdframed}

\hfill\newline
\begin{mdframed}[nobreak=true, backgroundcolor=lightgray!20] 
\begin{lstlisting}[style=SQL]
NULL <= 3 is evaluates to Unknown 
\end{lstlisting}
\end{mdframed}

\hfill\newline
\begin{mdframed}[nobreak=true, backgroundcolor=lightgray!20] 
\begin{lstlisting}[style=SQL]
SELECT St.Student_Name
FROM Students AS St 
WHERE NULL <= 20
\end{lstlisting}
\end{mdframed}

The above query will result to an empty set as for each row the WHERE clause will be evaluated to Unknown and as a consequence none of the rows will be appeared in the result. 

\hfill\newline
\begin{mdframed}[nobreak=true, backgroundcolor=lightgray!20] 
\begin{lstlisting}[style=SQL]
SELECT St.Student_Name
FROM Students AS St 
WHERE NULL = NULL
\end{lstlisting}
\end{mdframed}
Logically it is expected that the NULL equal NULL should be evaluate to true. Nevertheless, the WHERE clause is evaluate to Unknow for every row, and the result of this query is an empty set. 


\section{SQL standard issues} 

Below it is provided a few concrete examples which demonstrate that indeed some aspects of the Standard is implemented differently by each vendor. 
 
\noindent\textbf{Example:}

For example the SQL query below does not return identical results on both PostgreSQL and Oracle [14]. 


\begin{mdframed}[backgroundcolor=lightgray!20][h] 
Q1:SELECT * 
 \\FROM ( SELECT S.A, S.A FROM S ) R
\end{mdframed}

While it is expected Q1 to return identical results independently on which systems is executed on, this is not the case. It can be observed that Q1 will output a table with two columns named “A” in PostgreSQL. On the other hand, in Oracle database, the SQL query will return an compile-time error. Ιndisputably, there are differences between current DBMSs [1]. 
It can supposed that in most of the cases these differences are minor but if we take into account that these systems are used in many different fields, then we can quickly realise that small differences might be critical. The key idea is to be conducted an experimental evaluation of current DBMSs. 


\section{The Database Management Systems} 

Microsoft SQL Server: is a relational database management systems (RDBMS) developed by Microsoft. Microsoft offers different edition of its product such as Standard, Enterprise and Express. Express edition is a free edition of the SQL Server. It is mainly available on Windows. 

MySQL: is an open-source RDBMS which if freely available which is owned by Oracle. MySQL is used by huge companies such as Google, Facebook and Youtube. It is freely available on MacOS, Linux and Microsoft Windows 

PostgreSQL: is an open-source RDBMS which is trying to be SQL-Compliance and it focuses on portability.  It is freely available both on Microsoft Windows and Linux. 

IBM DB2: is a RDBMS developed by IBM and it is available on Linux and Microsoft Windows. 

Oracle Database: is a RDBMS developed by Oracle Corporation.  It available on MacOs, Linux and Microsoft Windows. 


 \begin{figure} 
      \centering
      \includegraphics[width=\textwidth,height=6cm]{Images/db_engines_chart}
      \caption{Popularity of modern DBMSs}
      \label{fig: Popularity of modern DBMSs}
    \end{figure}

The Figure 2.1 illustrates the most popular DBMSs from 2013 to 2017 using a calculated score based on the Google Trend, number of mentions of DBMSs on the web and discussion of the systems in Stackoverflow and DBA Stack Exchange. It can be seen that the popularity of Oracle database, MySQL and Microsoft SQL Server remain constant over the years with a relatively high score. In addition, there is a stable rise for PostgreSQl in the last three years. Lastly, IBM DB2 popularity has decreased slightly in the last year. 





\chapter{Methodology}

\section{Methodology}
 
This chapter describes the primary Methodology that is used to test the SQL-Compliance of current DBMSs. 

As it was mention in the first chapter, the key idea of the current research is to examine whether current DBMSs comply according to SQL standards and highlight the main differences among those systems. More precisely, it will be examined five popular systems such as MySQL, IBM DB2, Microsoft SQL Server, PostgreSQL and Oracle Database. Aside from that this research has as target  to provide a complete architecture that can be used to systematically inspect the SQL-Compliance as these systems have been evolved rapidly. The architecture is consisted by three different tools. More precisely, it has been implemented two different tools such as Random SQL Query Generator Tool and the Comparison Tool. The third one which is used to produce random realistic data realistic data is open-source tool namely Datafiller. The data can be generated based on a database schema which is given to the tools as a parameter. Complex data can be generated by specifying some parameter to the tool.  The purpose of the SQL generator tool is to generate a diversity of SQL queries by using SQL commands which are defined in the SQL standard. In that way, it can be examined whether answers to SQL queries are the same independently of which DBMS they are executed on. Therefore, the comparison tool is used to identify if the results are identical. A detailed explanation of the internal implementation and structure of those tools is given in the following Chapter. The implementation of the tool is in such a way that a new DBMS or functionality could be added without affecting the rest implementation.The comparison of DBMS’s results will be performed using a main-memory data structure for achieving efficiency. Differences among current DBMSs are automatically documented by the tool. For example, a query may not be executed on one of the DBMSs and raise an error. Hence, this error is logged into the log file that the comparison tool generates. 








 
\chapter{Implementation}
In this chapter it is described in details the framework that it has been implemented for assess the SQL-compliance for current DBMSs.

\section{Implementation}
This chapter describes the framework that is implemented for assessing the SQL-compliance of current DBMSs.

The framework is divided into three different tools as introduced in the previous chapter. The first two tools are implemented in the Java programming language and the third one is implemented in the Python programming language which is an open-source project that can generate random realistic data.  This project uses Java programming language since it is very popular and it can be used to build cross-platform systems. The complete framework consists of twenty java classes. 


 \begin{figure} 
      \centering
      \includegraphics[width=\textwidth,height=6cm]{Images/1-implemen_detail}
      \caption{Random SQL Generator Architecture}
      \label{fig:counting-methods}
  \end{figure}

\subsection{Random SQL generator engine}
An important component of the framework is the random SQL query generator tool which is used to generate random SQL queries for assessing modern DBMSs. The SQL language consists of numerous SQL commands and therefore some of them are simple in terms of their usage such as ‘SELECT...FROM...WHERE’ where some others are more trickier such as ‘GROUP BY’, ‘HAVING’ or aggregation functions such as MAX, ‘COUNT’ or AVG. These SQL commands need to be properly generated and syntactically correct in order to avoid syntax or semantic errors.. As a result, for generating meaningful and syntactically valid queries, the current implementation of the generator tool must take into consideration table and attribute Naming, data type compatibility and projection.

In addition, the tool has been designed in a modular way for reusability and therefore new systems can easily extend it without the need of changing the whole structure of the tool.

Moreover, an important decision that is taken is with regards to the internal implementation of the generator tool, in order to make it feasible to generate more valid and syntactically correct SQL queries. Hence, an internal representation is implemented with each of its classes responsible for generating a different SQL clause that contributes to the overall query. For example, one of the java classes generates the SELECT clause. Having different classes for each SQL statement makes it easier to extend the tool in order to add new functionality and at the same time there is no need to change other parts of the tool. The final SQL query is converted to an SQL string which then is executed to the current DBMSs with the contribution of the comparison tool. An important note is that it is not feasible to generate SQL strings directly because the attribute names and data types need to be tracked first. If SQL strings were to be used, it would not be possible to check if a variable mentioned in the ‘WHERE’ clause is also mentioned in the ‘FROM’ clause or if that variable comes as a parameter from an external query. Thus, there is a need to track attributes for each clause and in order to achieve that, LinkedList and HashMap data-structures were used.


\subsection{Configuration file}
The generator tool also includes a configuration file for partially controlling the random SQL queries and a detailed explanation is given as follows: The configuration file is used to provide information to the generator tool. Therefore, many parameters can be specified from the configuration file. Below, the format of the configuration file and subsequently a comprehensive explanation is given  for every parameter. 


\begin{mdframed}[backgroundcolor=gray!20] 
  This configuration file will be used to give various parameter
  \\ to the SQLEngine

  Maximum number of tables in the FROM STATEMENT
\\maxTablesFrom =3

  Maximum number of attributes in the SELECT STATEMENT
\\maxAttrSel =5

  Maximun number of comparisons in the WHERE STATEMENT
\\maxCondWhere =7

 Represents the probability of having constants or NULL comparison \\ in the WHERE STATEMENT
\\probWhrConst = 0.8

 nestLevel = 4
\end{mdframed}

Another important decision that should be taken into account is the provision of relations and attributes to the tool. An initial approach was to use them as parameters in the configuration file. Although, this approach works acceptably, the resulting tool lacks portability, especially when a DBMS has a large number of tables and columns. Hence, It will be time-consuming to give these parameters via the configuration file. Thus, an efficient approach is to retrieve the whole schema from DBMSs automatically. As a result, the implemented tool has the capability to automatically retrieve the whole schema for any DBMS just by providing the credential for connecting to the database in the configuration file.    

\hfill\newline
All the parameters are described as follow:
\begin{description}
   \item[$\bullet$ relations and attributes] =  parameters are used to provide to the generator tool the tables and columns for generating SQL queries according to the database schema. The current architecture has the capability to automatically retrieve the whole database schema from any DBMS by just providing the credentials in the configuration file such as user, pass and dbName parameters. 

\item[$\bullet$ MaxTablesFrom] = parameter is used to set an upper bound of the number of tables that a ‘FROM’ clause can have. If the upper bound is greater than the total number of tables in the schema, then the upper bound is automatically set to the total number of tables. 

\item[$\bullet$ MaxAttrSel] =  parameter indicates an upper bound of projections in the ‘SELECT’ clause. In other words, it is the total number of attributes that can be selected in the ‘SELECT’ clause.
 
\item[$\bullet$ MaxCondWhere]= parameter represents the total number of comparison that the ‘WHERE’ clause can have. 
 
\item[$\bullet$ ProbWhrConst] = parameter represents the probability of having comparisons with constant or ‘Null’ in the ‘WHERE’ clause. Therefore, a number between 0 and 1 can be given for this parameter.  

\item[$\bullet$ RepAlias]= parameter indicates the probability of having repetition of alias in the ‘SELECT’ clause. 

\item[$\bullet$ NestLevel]= parameter represents the maximum level of nesting that a query can have. For generating such a query many considerations should be taken into account. For example, we should track attributes for outer queries, because inner query can access outer attributes or attributes from its ‘FROM’ clause. The opposite is not true, meaning that we cannot access attributes from an inner query. 

\item[$\bullet$ ArithCompSel] = parameter represents the probability of having arithmetic comparisons in the ‘SELECT’ clause. 

\item[$\bullet$ Dinstinct ] = parameter represents the probability of the ‘DISTINCT’ command appearing in the SQL statement. 

\item[$\bullet$ StringInSel] = parameter indicates the probability of projecting an attribute of type String or having String functions in the ‘SELECT’ clause. 

\item[$\bullet$ StringInWhere] = parameter represents the probability of having String comparison in ‘WHERE’ clause. 
 
\end{description} 

It can be seen from the configuration file that many parameters of the random query generator can be controlled but, on the other hand, it does not imply that the diversity of SQL queries that can be generated is restricted. For example, an upper bound of tables that appear in the FROM clause can be set. In that way, having an enormous table size from cartesian product can be avoided.

In addition, it is not so common to have constant comparisons in an SQL query even though SQL Standard supports it. Thus, SQL queries which have constant comparisons are generated but only with a limited number based on the probability which is given in the configuration file.


\textbf{SQL Query generated by the random query generator} 

\begin{mdframed}[backgroundcolor=gray!20] 
SELECT r41.A AS A0
\\FROM r4 AS r41
\\WHERE 1 > 2
\end{mdframed}

\section{Comparison Tool} 
The generated SQL queries are evaluated on five different DBMSs as they were mentioned in the methodology chapter. The comparison tool (CT) takes into consideration all methods of accessing different databases so that the process of executing and comparing identical queries among all databases can be automated.

The CT is illustrated below and  a detailed explanation of its internal implementation is provided.
 \begin{figure} 
      \centering
      \includegraphics[width=\textwidth,height=6cm]{Images/2-ComparisonTool}
      \caption{Simple Architecture of Comparison Tool}
      \label{fig:counting-methods}
  \end{figure}

The tool is implemented in Java programming language and is one of the main components of the complete framework. Specifically, the CT is used to compare the results of each randomly generated query. As each DBMS uses various algorithms to evaluate each query, sometimes the results are identical but the order or format differs. As a consequence, for efficient comparison of the query results between DBMSs, the following approach is used: Initially, the query results are stored in a LinkedList, so that an efficient in-memory sorting algorithm can be performed, called Quicksort. In that way, the results among the DBMSs should be the same and a row by row comparison is performed to verify that they are identical. If a difference is found or a query raises an error, on one of the tested DBMSs, then a detailed explanation of the difference or the error message is recorded in the log file.


An important consideration for implementing this tool is its ease of reuse and extensibility by other systems or software. As each system needs its own credentials such as username, password and database name, a configuration file is created for providing these information to the CT. By doing so, the CT becomes portable and adaptable and a new system can be easily added. As shown in the experiments chapter, our approach can detect many important differences and issues. 

 \begin{figure} 
      \centering
      \includegraphics[width=\textwidth,height=6cm]{Images/3-ComparisonTool}
      \caption{Architecture of Comparison Tool}
      \label{fig:counting-methods}
  \end{figure}

\section{Random generated data}

Datafiller is a well-known open-source project that provides the capability of generating random data [3]. For our project, Datafiller will be used for generating a diversity of  database instances in order to evaluate all major DBMSs. More precisely, the Datafiller script generates random data, based on a an SQL  schema which is provided as a parameter and it takes into account constraints of that schema for generating valid data. For example, it takes into account the domain of each field and if each field should be unique, foreign key or primary key. Another important parameter is the df: null=rate\% which indicates the nullable rates. It necessary to test the behaviour of current DBMS in databases with nulls in order to evaluate whether all DBMSs behave in a similar fashion regarding null values.     

Additionally, more complex parameters can be provided as well, such as the number of tuples per table using --size SIZE parameter. It is worth mentioning that these parameters should be defined within the schema script and should start with ‘-- df’.  Furthermore, we can generate more realistic data by providing some information in the schema SQL script. For instance, if there is a field which represents a date, then we can provide a range in order for the Datafiller to generate dates only within that range. This can be achieved by specifying the following parameter: range -- df: start=year-month-day end=year-month-day next to the Date field. Subsequently, we need to add the --filter parameter while running the script. These are only some of the important parameters that the Datafiller provides but apart from these, it provides more sophisticated properties which are out of importance for our project.



 \begin{figure} 
      \centering
      \includegraphics[width=\textwidth,height=2cm]{Images/4-Datafiller}
      \caption{Method of importing csv files}
      \label{fig:counting-methods}
  \end{figure}  


Datafiller supports data importing only into PostgreSQL and therefore, for our project we need to import the random generated data into five DBMSs. For this reason, we transform the data into CSV format, as all of them support importing data from CSV files.  


\chapter{Experimental Evaluation}
This chapter presents the procedure that it is followed in order to provide the experimental evidences. Subsequently, It is presented the main findings with appropriate explanation according to the Standard. Also, it is described  the environment and the DBMSs that they have been tested. 

\section{The experiment Set-Up}
The evaluation is carried out on Windows 8 with i7 CPU, 12GB Ram and a solid state disk (SSD). Also, the  following versions of DBMSs have been installed: PostgreSQL Version 9.6, Microsoft SQL Server Express Edition 2016, IBM Db2 Express-C, Oracle Database 12c and  MySQL Community Edition 5.7. 
Apart of generating huge numbers of SQL queries, it is also checked if new findings can be emerged by conducting the experiments in different schemas with different data types and numbers of relations. As a result, it is checked the following data types:  TEXT, CHAR, VARCHAR, INTEGER, SMALLINT, BIGINT. Moreover, the experiments are conducted in different schemas varying from two to ten relations with two to ten attributes. In addition, it is generated data using different rate of NULLs such as 5%, 10%, 20% and 40% for identify if differences can arise depending on the number of NULLs that a database contains.  Furthermore, the size of instances varied from two to one thousand rows which were generating using data fiiller. Having huge instances decrease the throughput of queries that can be executed per second and thus less queries will be checked. Taking into account that issues and different interpretations can be raised even in small instances, it is end up realized that small instances and schema can still reveal crucial differences. 

\section{Experiment Results}
The implemented architecture is used to conduct the experiments and in that way it is also checked whether the current implementation if is capable to detect any differences among modern DBMSs. The results are provided below with the following format: Firstly,  It is presented  the SQL query that cause an error or a semantic issue on one or more systems. Meaning that some features may not be implemented by all systems or use a different commands or interpret it differently.  Then, a table for each query is provided that demonstrates a raised error of any system, otherwise, the keyword ‘works’ meaning that the SQL query is executed without to raise any error. Thereafter, a comprehensive explanation is given for each problematic query by explaining about the source of the problem and giving an explanation according to the SQL standard.   

\subsection{Dif 1}

\textbf{Q1:}
\begin{mdframed}[nobreak=true, backgroundcolor=lightgray!20] 
\begin{lstlisting}[style=SQL]
SELECT r41.A AS A0
FROM r4 AS r41
WHERE 1*1
\end{lstlisting}
\end{mdframed}

\begin{table}[h]
\centering
\caption{Difference 1}
\begin{tabular}{|p{2cm}|p{11.5cm}| }
\hline
\textbf{DBMS} & \textbf{Result Message}                                                                                                 \\ \hline
Mysql         & Works                                                                                                                   \\ \hline
PostgreSQL    & {[}42804{]} ERROR: argument of WHERE must be type boolean, not type integer                                             \\ \hline
MS Server     & {[}S0001{]}{[}4145{]} An expression of non-boolean type specified in a context where a condition is expected, near '1'. \\ \hline
Oracle        & {[}42000{]}{[}933{]} ORA-00933: SQL command not properly ended                                                          \\ \hline
IBM DB2       & Works                                                                                                                   \\ \hline
\end{tabular}
\end{table}

According to SQL standard each expression in the WHERE clause should be evaluated to a boolean type such as True or False. Thus, DBMSs evaluate each row based on the SQL query and if the WHERE clause is evaluated to True, then the specific row appears in the output results. We expect that arithmetic comparisons between two numbers should return an integer type instead of boolean type. Thus, if the query 1 is executed on any DBMSs should raise an error. Nevertheless, Mysql and IBM DB2 execute the query without to raise any error even though there is an arithmetic comparison  in the WHERE clause such as 1*1. Albeit expressions in the WHERE clause should return a boolean type, these two DBMSs convert arithmetic comparison into a boolean type. On the contrary, the rest three DBMSs throw an exception while they executing the query which is something reasonable as they expect a boolean type in the WHERE Clause. 

\subsection{Dif 2}
  
\textbf{Q2:}
\begin{mdframed}[backgroundcolor=lightgray!20] 
\begin{lstlisting}[style=SQL]
SELECT r21.A AS A0 
FROM r2 AS r21
WHERE true
\end{lstlisting}
\end{mdframed}
 
 
\begin{table}[h]
\centering
\caption{Difference 2}
\label{my-label}
\begin{tabular}{|p{2cm}|p{11.5cm}| }
\hline
\textbf{DBMS} & \textbf{Result Message}                                                                                                   \\ \hline
Mysql         & Works                                                                                                                     \\ \hline
PostgreSQL    & Works                                                                                                                     \\ \hline
MS Server     & {[}S0001{]}{[}4145{]} An expression of non-boolean type specified in a context where a condition is expected, near 'true' \\ \hline
Oracle        & {[}42000{]}{[}920{]} ORA-00920: invalid relational operator                                                               \\ \hline
IBM DB2       & Works                                                                                                                     \\ \hline
\end{tabular}
\end{table}

As it was mentioned before each expression in the WHERE clause is evaluated to true or false. Thus, instead of having an expression, it can be specified the true/false keyword which is a boolean type. As a consequence, it will be reasonable to have the ‘true’ keyword in the WHERE clause meaning that each row will be included in the results. Nevertheless, not all the DBMSs support this. MS server and Oracle do not support this keyword. With respect to SQL standard there is not explicit mention about the keyword true in the WHERE Clause. 

\hfill\newpage
\subsection{Dif 3}

\textbf{Q3:}
\begin{mdframed}[backgroundcolor=lightgray!20]
\begin{lstlisting}[style=SQL]
SELECT  NULL/NULL, 1/2, NULL-NULL
FROM r2 AS  r21, r4  AS  r41
WHERE  r41.B > r21.B 
\end{lstlisting}
\end{mdframed}

\begin{table}[h]
\centering
\caption{Difference 3}
\label{my-label}
\begin{tabular}{|p{2cm}|p{11.5cm}| }
\hline
\textbf{DBMS} & \textbf{Result Message}                                                                                                                                   \\ \hline
Mysql         & Works                                                                                                                                                     \\ \hline
PostgreSQL    & {[}42725{]} ERROR: operator is not unique: unknown / unknown Hint: Could not choose a best candidate operator. You might need to add explicit type casts. \\ \hline
MS Server     & Works                                                                                                                                                     \\ \hline
Oracle        & Works                                                                                                                                                     \\ \hline
IBM DB2       & Works                                                                                                                                                     \\ \hline
\end{tabular}
\end{table}

According to SQL standards boolean data type comprises the distinct truth values True and False. Apart from these values, boolean data type supports the truth value Unknown as the NULL value. As a result, the SQL standard does not make a distinction between NULL value of the boolean data type and the truth value Unknown. It can be seen from the above query that the answer of  the expression NULL/NULL will result to the value Unknown. It worthy mentioning that some DBMSs represent Unknown value as NULL. Nevertheless, the specific expression throw an exception in PostgreSQL, while to the rest DBMSs is executed without any error. 

\subsection{Dif 4}
  
\textbf{Q4:}
\begin{mdframed}[backgroundcolor=lightgray!20]
\begin{lstlisting}[style=SQL]
SELECT  r21.B/3
FROM  r2 AS r21, r4 AS r41
WHERE NOT(NOT(r41.A <> 18 ) )  
\end{lstlisting}
\end{mdframed}

 
\begin{table}[h]
\centering
\caption{Difference 4}
\label{my-label}
\begin{tabular}{|p{2cm}|p{11.5cm}| }
\hline
\textbf{DBMS} & \textbf{Result Message} \\ \hline
Mysql         & Works                   \\ \hline
PostgreSQL    & Works                   \\ \hline
MS Server     & Works                   \\ \hline
Oracle        & Works                   \\ \hline
IBM DB2       & Works                   \\ \hline
\end{tabular}
\end{table}

Even though the Q4 is executed without any error in the current DBMSs, the results differ slightly in terms of their return type. In DB2, PostgreSQL and MS Server the results for the column  r21.B/3  are returned as integer type where on the contrary with MySQL and Oracle db where the results are returned as decimal type. 

\subsection{Dif 5}
  
\textbf{Q5:}
\begin{mdframed}[backgroundcolor=lightgray!20]
\begin{lstlisting}[style=SQL]
SELECT (MIN(r41.B) % AVG(r41.A) ) 
FROM r4 AS r41
WHERE (10 >= 19 )     
GROUP BY r41.A, r41.B
\end{lstlisting}
\end{mdframed}

\begin{table}[h]
\centering
\caption{Difference 5}
\label{my-label}
\begin{tabular}{|p{2cm}|p{11.5cm}| }
\hline
\textbf{DBMS} & \textbf{Result Message}                                \\ \hline
Mysql         & Works                                                  \\ \hline
PostgreSQL    & Works                                                  \\ \hline
MS Server     & Works                                                  \\ \hline
Oracle        & {[}22019{]}{[}911{]} ORA-00911: invalid ‘\%’ character \\ \hline
IBM DB2       & Works                                                  \\ \hline
\end{tabular}
\end{table}

It worthy mentioning that arithmetic operations such as addition, multiplications and  subtraction are supported in the SELECT clause. Another important arithmetic operation which is also  supported in the SELECT Clause is the.

\hfill\newpage
\subsection{Dif 6}
  
\textbf{Q6:}
\begin{mdframed}[backgroundcolor=lightgray!20]
\begin{lstlisting}[style=SQL]
SELECT r41.A AS A0
FROM r4 AS r41
\end{lstlisting}
\end{mdframed}

\begin{table}[h]
\centering
\caption{Difference 6}
\label{my-label}
\begin{tabular}{|p{2cm}|p{11.5cm}| }
\hline
\textbf{DBMS} & \textbf{Result Message}                                        \\ \hline
Mysql         & Works                                                          \\ \hline
PostgreSQL    & Works                                                          \\ \hline
MS Server     & Works                                                          \\ \hline
Oracle        & {[}42000{]}{[}933{]} ORA-00933: SQL command not properly ended \\ \hline
IBM DB2       & Works                                                          \\ \hline
\end{tabular}
\end{table}

With respect to SQL standards alias names can be used both with attributes in the SELECT Clause and for tables in the FROM clause. For defining an alias the keyword ‘AS’ is used. The idea of alias is for renaming relations and columns of the results in order to make them more readable. In addition, it can renamed a subquery in the FROM Clause and subsequently It can be accessed using its alias. Even though almost all DBMSs support alias in the FROM AND SELECT clause, Oracle’s db  allows to use AS when defining column aliases, but it does not allow you to use AS when defining table aliases 


\subsection{Dif 7}
  
\textbf{Q7:}
\begin{mdframed}[backgroundcolor=lightgray!20]
\begin{lstlisting}[style=SQL]
(SELECT r41.A AS A0
 FROM r4 AS r41 ) 
EXCEPT ALL
(SELECT r21.A AS A0
 FROM r2 AS r21, r4 AS r42 )
\end{lstlisting}
\end{mdframed}

\begin{table}[h]
\centering
\caption{difference 7}
\label{my-label}
\begin{tabular}{|p{2cm}|p{11.5cm}| }
\hline
\textbf{DBMS} & \textbf{Result Message}                                                                                                                                                 \\ \hline
Mysql         & {[}42000{]}{[}1064{]} You have an error in your SQL syntax; check the manual that,corresponds to your MySQL server version for the right syntax to use near 'EXCEPT ALL \\ \hline
PostgreSQL    & Works                                                                                                                                                                   \\ \hline
MS Server     & S0002{]}{[}324{]} The 'ALL' version of the EXCEPT operator is not supported.                                                                                            \\ \hline
Oracle        & {[}42000{]}{[}933{]} ORA-00933: SQL command not properly ended                                                                                                          \\ \hline
IBM DB2       & Works                                                                                                                                                                   \\ \hline
\end{tabular}
\end{table}

\hfill\\\\
EXCEPT ALL is an optional feature which is defined in the SQL standard and result to return all rows from the outer relation which are not present in the inner relation without removing the duplicates. Thus, this operator is fully supported by the SQL standard but it is optional.  It can be seen from the above table that SQL server, MySql and Oracle db do not support this operator. Nevertheless, regarding oracle db, it is important mentioning that it supports the same operator but with different name which is the ‘MINUS ALL’ keyword. Thus, we can use the MINUS ALL rather than EXCEPT ALL which has exactly the same behaviour.   

\subsection{Dif 8}
  
\textbf{Q8:}
\begin{mdframed}[backgroundcolor=lightgray!20]
\begin{lstlisting}[style=SQL]
(SELECT r41.A AS A0
 FROM r4 AS r41, r2 AS r21, r3 AS r31
 WHERE NOT(r31.B <> r41.A ) )
EXCEPT
(SELECT r21.A AS A0
 FROM r2 AS r21, r4 AS r41
 WHERE r41.A <> r41.B )
\end{lstlisting}
\end{mdframed}


\begin{table}[h]
\centering
\caption{Difference 8}
\label{my-label}
\begin{tabular}{|p{2cm}|p{11.5cm}|}
\hline
\textbf{DBMS} & \textbf{Result Message}                                                                                                                                             \\ \hline
Mysql         & {[}42000{]}{[}1064{]} You have an error in your SQL syntax; check the manual that corresponds to your MySQL server version for the right syntax to use near 'EXCEPT \\ \hline
PostgreSQL    & Works                                                                                                                                                               \\ \hline
MS Server     & Works                                                                                                                                                               \\ \hline
Oracle        & {[}42000{]}{[}933{]} ORA-00933: SQL command not properly ended                                                                                                      \\ \hline
IBM DB2       & Works                                                                                                                                                               \\ \hline
\end{tabular}
\end{table}

\hfill\\\\\\
Except is a mandatory feature in the  SQL standard. Using this feature results to return all rows from the outer relation which are not present in the inner relation with removing the duplicates.  Oracle instead of supports EXCEPT use ‘MINUS’ which has a similar behavior. On the contrary, MySql does not support EXCEPT at all even though it is a compulsory feature according to the SQL standard. The rest DBMSs support this operator. 

\subsection{Dif 9}
  
\textbf{Q9:}
\begin{mdframed}[backgroundcolor=lightgray!20]
\begin{lstlisting}[style=SQL]
(SELECT r41.A AS A0
 FROM r4 AS  r41, r3 AS r31
 WHERE (NULL <= 6 OR NOT(r31.B <> r41.A ) ) )
INTERSECT ALL
(SELECT r21.A AS A0
 FROM r2 AS r21, r4 AS r41
 WHERE r41.A <> r41.B OR ( 0 <> 14)  AND r41.A > r21.B)
\end{lstlisting}
\end{mdframed}

 
\begin{table}[h]
\centering
\caption{Difference 9}
\label{my-label}
\begin{tabular}{|p{2cm}|p{11.5cm}| }
\hline
\textbf{DBMS} & \textbf{Result Message}                                                                                                                                          \\ \hline
Mysql         & {[}42000{]}{[}1064{]} You have an error in your SQL syntax; check the manual that corresponds to your MySQL server version for the right syntax to use near 'ALL \\ \hline
PostgreSQL    & Works                                                                                                                                                            \\ \hline
MS Server     & {[}S0001{]}{[}324{]} The 'ALL' version of the INTERSECT operator is not supported.                                                                               \\ \hline
Oracle        & {[}42000{]}{[}928{]} ORA-00928                                                                                                                                   \\ \hline
IBM DB2       & Works                                                                                                                                                            \\ \hline
\end{tabular}
\end{table}

\hfill\newpage
With respect to the SQL standard INTERSECT ALL is an optional feature and it should return all rows which are presented in both inner and outer queries results and without removing duplicates. It can be seen from the above table that the feature is only implemented in PostgreSQL and DB2.  
 
\subsection{Dif 10}
  
\textbf{Q10:}
\begin{mdframed}[backgroundcolor=lightgray!20]
\begin{lstlisting}[style=SQL]
SELECT  7/0 AS ART0, 1%NULL AS ART1
FROM r1 AS r11
WHERE (NULL = 19 AND r11.b <> 5)   
\end{lstlisting}
\end{mdframed}
  
\begin{table}[h]
\centering
\caption{Difference 10}
\label{my-label}
\begin{tabular}
{|p{2cm}|p{11.5cm}| }
\hline
\textbf{DBMS} & \textbf{Result Message}                                 \\ \hline
Mysql         & Works                                                   \\ \hline
PostgreSQL    & {[}22012{]} ERROR: division by zero                     \\ \hline
MS Server     & {[}S0001{]}{[}8134{]} Divide by zero error encountered. \\ \hline
Oracle        & ORA-01476: divisor is equal to zero                     \\ \hline
IBM DB2       & {[}22012{]}{[}-801{]} Division by zero was attempted..  \\ \hline
\end{tabular}
\end{table}

Apparently, a division with zero should always throw an error. Nevertheless, MySQL does not throw any error and the result of the division with zero is NULL. 

 
\subsection{Dif 11}

\textbf{Q11:}
\begin{mdframed}[backgroundcolor=lightgray!20]
\begin{lstlisting}[style=SQL]
SELECT r11.a AS A1, r11.b AS A2
FROM r1 AS r11
WHERE (r11.b, r11.a) IN (SELECT r12.a AS A4, r12.b AS A3
                         		 FROM r1 AS r12)
\end{lstlisting}
\end{mdframed}

\begin{table}[h]
\centering
\caption{Difference 11}
\label{my-label}
\begin{tabular}{|p{2cm}|p{11.5cm}|}
\hline
\textbf{DBMS} & \textbf{Result Message}                                                                                                 \\ \hline
Mysql         & Works                                                                                                                   \\ \hline
PostgreSQL    & Works                                                                                                                   \\ \hline
MS Server     & {[}S0001{]}{[}4145{]} An expression of non-boolean type specified in a context where a condition is expected, near ','. \\ \hline
Oracle        & Works                                                                                                                   \\ \hline
IBM DB2       & Works                                                                                                                   \\ \hline
\end{tabular}
\end{table}


The Q11 will not run on MS Server. The problem is that the ‘IN’ operator does not accept more than one attribute and as a result the DBMS will throw an exception. In the rest DBMS the query is working properly. 

\subsection{Dif 12}
 
\textbf{Q12:}
\begin{mdframed}[backgroundcolor=lightgray!20]
\begin{lstlisting}[style=SQL]
SELECT  r31.b AS A1
FROM r3 AS r31
WHERE r31.a >= r31.a
GROUP BY r31.a
\end{lstlisting}
\end{mdframed}
 
\begin{table}[h]
\centering
\caption{Difference 12}
\label{my-label}
\begin{tabular}{|p{2cm}|p{11.5cm}| }
\hline
\textbf{DBMS} & \textbf{Result Message}                                                                                                                                                                                                                                                                 \\ \hline
Mysql         & Works                                                                                                                                                                                                                                                                                   \\ \hline
PostgreSQL    & {[}42803{]} ERROR: column "r31.b" must appear in the GROUP BY clause or be used in an aggregate function Position: 9                                                                                                                                                                    \\ \hline
MS Server     & {[}S0001{]}{[}8120{]} Column 'r3.B' is invalid in the select list because it is not contained in either an aggregate function or the GROUP BY clause.                                                                                                                                   \\ \hline
Oracle        & {[}42000{]}{[}979{]} ORA-00979: not a GROUP BY expression                                                                                                                                                                                                                               \\ \hline
IBM DB2       & {[}42803{]}{[}-119{]} An expression starting with "B" specified in a SELECT clause, HAVING clause, or ORDER BY clause is not specified in the GROUP BY clause or it is in a SELECT clause, HAVING clause, or ORDER BY clause with a column function and no GROUP BY clause is specified \\ \hline
\end{tabular}
\end{table}

\hfill\newline
\subsection{Dif 13}
 
\textbf{Q13:}
\begin{mdframed}[backgroundcolor=lightgray!20]
\begin{lstlisting}[style=SQL]
SELECT  NULL+NULL AS ART1
FROM r4 AS r41, r5 AS r51
WHERE  r41.a >= r41.a OR ( r41.a >= 4 )
GROUP BY r41.a
HAVING MIN(r41.a) < 7506
\end{lstlisting}
\end{mdframed}

\begin{table}[h]
\centering
\caption{Difference 13}
\label{my-label}
\begin{tabular}{|p{2cm}|p{11.5cm}| }
\hline
\textbf{DBMS} & \textbf{Result Message}                                                                                                                                  \\ \hline
Mysql         & Works                                                                                                                                                    \\ \hline
PostgreSQL    & {[}42725{]} ERROR: operator is not unique: unknown + unknown Hint: Could not choose a best candidate operator. You might need to add explicit type casts \\ \hline
MS Server     & Works                                                                                                                                                    \\ \hline
Oracle        & Works                                                                                                                                                    \\ \hline
IBM DB2       & Works                                                                                                                                                    \\ \hline
\end{tabular}
\end{table}

\hfill\newpage
\subsection{Dif 14}

\textbf{Q14:}
\begin{mdframed}[backgroundcolor=lightgray!20]
\begin{lstlisting}[style=SQL]
SELECT  'SQL' || 'STANDARD'
FROM R1
\end{lstlisting}
\end{mdframed}
 
\begin{table}[h]
\centering
\caption{Difference 14}
\label{my-label}
\begin{tabular}{|p{2cm}|p{11.5cm}| }
\hline
\textbf{DBMS} & \textbf{Result Message}                         \\ \hline
Mysql         & Works                                           \\ \hline
PostgreSQL    & Works                                           \\ \hline
MS Server     & {[}S0001{]}{[}102{]} Incorrect syntax near '|'. \\ \hline
Oracle        & Works                                           \\ \hline
IBM DB2       & Works                                           \\ \hline
\end{tabular}
\end{table}

According to the SQL standard concatenation should be supported by DBMSs with a double-pipe mark ‘||’ and the purpose is to concatenate two or more strings into one. Thus, executing Q14 is expected the result to be ‘SQLSTANDARD’. Nevertheless, MySQL supports the double pipe operator, but it treats the double-pipe “||” as a logical OR and thus the query returns 0. For accomplishing concatenation in MySQL, it uses the CONCAT() function which as parameter one or more strings.   Microsoft SQL Server does not support this operator and raise an error and instead of this operator, it uses a different one such as ‘+’  in order to perform the concatenation. Oracle and PostgreSQL support the double-pipe  concatenation operator. 



\subsection{Dif 15}
\textbf{Q15:}
\begin{mdframed}[backgroundcolor=lightgray!20]
\begin{lstlisting}[style=SQL]
SELECT  *
FROM  r1 AS R1 , r1 AS R1
\end{lstlisting}
\end{mdframed}

\begin{table}[h]
\centering
\caption{Difference 15}
\label{my-label}
\begin{tabular}{|p{2cm}|p{11.5cm}| }
\hline
\textbf{DBMS} & \textbf{Result Message}                                                                       \\ \hline
Mysql         & {[}42000{]}{[}1066{]} Not unique table/alias: 'R1'                                            \\ \hline
PostgreSQL    & {[}42712{]} ERROR: table name "r1" specified more than once                                   \\ \hline
MS Server     & {[}S0001{]}{[}1011{]} The correlation name 'R1' is specified multiple times in a FROM clause. \\ \hline
Oracle        & {[}42000{]}{[}933{]} ORA-00933: SQL command not properly ended                                \\ \hline
IBM DB2       & Works                                                                                         \\ \hline
\end{tabular}
\end{table}

It can be seen from Q15 that we have two times the same table with the same alias. It is expected that every DBMSs will raise an error while evaluating this query. Nevertheless, the query is executed in IBM DB2 database. 

\subsection{Dif 16}

\textbf{Q16:}
\begin{mdframed}[backgroundcolor=lightgray!20]
\begin{lstlisting}[style=SQL]
SELECT TRIM('    SQLSTANDARD    ')
FROM R1;
\end{lstlisting}
\end{mdframed}

\begin{table}[h]
\centering
\caption{Difference 16}
\label{my-label}
\begin{tabular}{|p{2cm}|p{11.5cm}| }
\hline
\textbf{DBMS} & \textbf{Result Message}                                                  \\ \hline
Mysql         & Works                                                                    \\ \hline
PostgreSQL    & Works                                                                    \\ \hline
MS Server     & {[}S00010{]}{[}195{]} 'TRIM' is not a recognized built-in function name. \\ \hline
Oracle        & Works                                                                    \\ \hline
IBM DB2       & Works                                                                    \\ \hline
\end{tabular}
\end{table}

\hfill\newpage
According to SQL standard trim function return the string which is given as argument with leading and/or trailing pad character. This function is supported by most of DBMSs, except MS Server. 


\subsection{Dif 17}

\textbf{Q17:}
\begin{mdframed}[backgroundcolor=lightgray!20]
\begin{lstlisting}[style=SQL]
SELECT  2 * 5 AS ART
WHERE ( 1 = 1 )
WHERE r1.c3 LIKE 'standard%'
\end{lstlisting}
\end{mdframed}


\begin{table}[h]
\centering
\caption{Difference 17}
\label{my-label}
\begin{tabular}{|p{2cm}|p{11.5cm}| }
\hline
\textbf{DBMS} & \textbf{Result Message}                                                                                                                                                     \\ \hline
Mysql         & {[}42000{]}{[}1064{]} You have an error in your SQL syntax; check the manual that corresponds to your MySQL server version for the right syntax to use near 'WHERE (1 = 1 ) \\ \hline
PostgreSQL    & Works                                                                                                                                                                       \\ \hline
MS Server     & Works                                                                                                                                                                       \\ \hline
Oracle        & {[}42000{]}{[}923{]} ORA-00923: FROM keyword not found where expected                                                                                                       \\ \hline
IBM DB2       & {[}42601{]}{[}-104{]} Expected tokens may include: "FROM"                                                                                                                   \\ \hline
\end{tabular}
\end{table}

\hfill\newpage
\subsection{Dif 18}


\textbf{Q18:}
\begin{mdframed}[backgroundcolor=lightgray!20]
\begin{lstlisting}[style=SQL]
SELECT  SUBSTRING ('Standard', 1, 4)
FROM  R1
\end{lstlisting}
\end{mdframed}

\begin{table}[h]
\centering
\caption{My caption}
\label{my-label}
\begin{tabular}{|p{2cm}|p{11.5cm}| }
\hline
\textbf{DBMS} & \textbf{Result Message}                                         \\ \hline
Mysql         & Works                                                           \\ \hline
PostgreSQL    & Works                                                           \\ \hline
MS Server     & Works                                                           \\ \hline
Oracle        & {[}42000{]}{[}904{]} ORA-00904: "SUBSTRING": invalid identifier \\ \hline
IBM DB2       & Works                                                           \\ \hline
\end{tabular}
\end{table}

The Substring function is defined in the SQL standard as an optional feature. Mysql, PostgreSQL, IBM DB2 and Microsoft Sql Server support this function. Oracle db use Substr instead of Substring which has a similar behaviour. The prototype of oracle’s function is as follow: substr $(column_name, start_pos , no_of_characters)$. 


\subsection{Dif 19}

\textbf{Q19:}
\begin{mdframed}[backgroundcolor=lightgray!20]
\begin{lstlisting}[style=SQL]
SELECT "SQLSTANDARD"
FROM R1
\end{lstlisting}
\end{mdframed}

 
\begin{table}[h]
\centering
\caption{Difference 19}
\label{my-label}
\begin{tabular}{|p{2cm}|p{11.5cm}| }
\hline
\textbf{DBMS} & \textbf{Result Message}                                                        \\ \hline
Mysql         & Works                                                                          \\ \hline
PostgreSQL    & {[}42703{]} ERROR: column "SQLSTANDARD" does not exist Position: 8             \\ \hline
MS Server     & {[}S0001{]}{[}207{]} Invalid column name 'SQLSTANDARD'.                        \\ \hline
Oracle        & {[}42000{]}{[}904{]} ORA-00904: "SQLSTANDARD": invalid identifier              \\ \hline
IBM DB2       & {[}56098{]}{[}-727{]} An error occurred during implicit system action type "2" \\ \hline
\end{tabular}
\end{table}

\hfill\newpage
According to the SQL standard encompass by ‘. It is demonstrated in Q19 that MySQL allow a string to be encompassed by both ‘ and “ which make the SQL code less portable as the rest DBMSs raise an error if it is used “ instead if ‘. 


\subsection{Dif 20}

\textbf{Q20:}
\begin{mdframed}[backgroundcolor=lightgray!20]
\begin{lstlisting}[style=SQL]
SELECT *
FROM r1 AS R1
WHERE R1.c3 = ’’’
\end{lstlisting}
\end{mdframed}
 
\begin{table}[h]
\centering
\caption{Difference 20}
\label{my-label}
\begin{tabular}{|p{2cm}|p{11.5cm}| }
\hline
\textbf{DBMS} & \textbf{Result Message} \\ \hline
Mysql         & Works                   \\ \hline
PostgreSQL    & Works                   \\ \hline
MS Server     & Works                   \\ \hline
Oracle        & Works                   \\ \hline
IBM DB2       & Works                   \\ \hline
\end{tabular}
\end{table}


Even though the above query is executed correctly on the DBMSs the semantic differs. As it was mentioned in the background chapter and more precisely in the missing value, Oracle database treats the empty string as NULL, on the contrary with the rest DBMSs which treats it as a normal string. Thus, all the comparisons in the WHERE clause involving NULL are evaluated to Unknown which means that the result will be empty as none of the rows will be satisfied.  On the other hand, if there is at least a row which contains an empty string by executing the above query will be returned in the result. It can be conclude that the above query will return all the rows that contain an empty string in the attribute c3 of the relation R1 in all the DBMSs except Oracle DB, where the result for this database will be empty.


\subsection{Dif 21}

\textbf{Q21:}
\begin{mdframed}[backgroundcolor=lightgray!20]
\begin{lstlisting}[style=SQL]
SELECT *
FROM R1
WHERE r1.c3 LIKE 'standard%'
\end{lstlisting}
\end{mdframed}


\begin{table}[h]
\centering
\caption{Difference 21}
\label{my-label}
\begin{tabular}{|p{2cm}|p{11.5cm}| }
\hline
\textbf{DBMS} & \textbf{Result Message} \\ \hline
Mysql         & Works                   \\ \hline
PostgreSQL    & Works                   \\ \hline
MS Server     & Works                   \\ \hline
Oracle        & Works                   \\ \hline
IBM DB2       & Works                   \\ \hline
\end{tabular}
\end{table}

Even though the above Query q21 does not raise an error in the current DBMS, it does not return the same results. All the tested systems contain a database that stores a tuple with the word STANDARD (in capital letters) in the attribute c3 of the relation r1. By executing this query, it can be observed that of the DBMSs are case sensitive with the LIKE operator. More precisely, it can be seen that executing this query both PostgreSQL and Oracle are case-sensitive, resulting to return an empty set. On the contrary with the rest systems which returns one row which is the row that contains the word standard in the attribute c3.  


\section{Summarize features}
The below table summarizes the main features of SQL language and illustrated which of them are not supported by all popular DBMSs. These findings have been discovered by conducting experiments using the random generator query tool and the comparison tool. We have generated a huge number of SQL queries in order to identify lot of cases where DBMSs behave differently. It is worthy mentioning that the process of conducting experiments is fully automated and in case where a difference is found, it is recorded in a log file with some useful explanation.    



\begin{table}[h]
\centering
\caption{My caption}
\label{my-label}
\begin{tabular}{|l|l|l|l|l|l|}
\hline
Operation     & Mysql & PostgreSQL & MS Server & Oracle & IBM DB2 \\ \hline
INTERSECT ALL &  \multicolumn{1}{c|}{\text{\sffamily X}}     &    \multicolumn{1}{c|}{\ding{52}}      &           &        &         \\ \hline
AS in FROM    &       &            &           &        &         \\ \hline
EXCEPT ALL    &       &            &           &        &         \\ \hline
              &       &            &           &        &         \\ \hline
\end{tabular}
\end{table}



\chapter{Conclusions}

\section{Conclusions}

In this project an entire framework is implemented composed by the random query generator tool and the comparison tool which are used to evaluate the SQL-compliance of five DBMSs. Also, we verified the correctness of the implemented tools by conducting the experiments and we demonstrated from the experiment evaluation chapter that the implemented tools are competent to reveal crucial differences and issues among current systems. Without a similar framework, it would be almost impossible to detect some of the differences and issues  by generating queries manually, or by testing all DBMSs empirically. Furthermore, as described in the related work, there is no similar framework, except of some documentations provided by the vendors of such systems and some other studies which presented some issues according to the Standard without having a systematic tool. 

In addition, a summarized table is provided exposing all the issues and incompatibilities between the most popular DBMSs which are of major importance for vendors, users, programmers and researchers of such systems. We believe that this project has contributed in the database management systems research area by introducing a new way of testing the SQL-Compliance of any DBMS and reporting any issues and incompatibilities that may arise. In addition, the implemented tools can be major importance for future vendors or researchers.

Furthermore, demonstrating and analyzing these incompatibilities makes users and programmers and researchers aware for these issues. We verified our assumptions that some parts of the Standard are implemented differently but it was somewhat surprising that so many differences have emerged. Lastly, the implemented framework is portable and can be extended efficiently. For example, although experimental evidences are provided for both numeric and alphanumeric data types, the random generator tool is implemented in such a way that can track any data types such as Date. In that way, it can be extended efficiently to generate queries with attributes of date as data type. 

\section{Summary of the findings}
 
The below table summarizes the main features of SQL language and illustrated which of them are not supported by all popular DBMSs. These findings have been discovered by conducting experiments using the random generator query tool and the comparison tool. We have generated a huge number of SQL queries in order to identify lot of cases where DBMSs behave differently. It is worthy mentioning that the process of conducting experiments is fully automated and in case where a difference is found, it is recorded in a log file with some useful explanation.    


\begin{table}[h]
\caption{Summarize results}
\label{my-label}
\resizebox{\columnwidth}{!}{%
\begin{tabular}{|l|l|l|l|l|l|}  
\hline
\textbf{Operation} & \textbf{Mysql} & \textbf{PostgreSQL} & \textbf{Microsoft SQL Server} & \textbf{Oracle} & \textbf{IBM DB2} \\ \hline
\textbf{INTERSECT ALL} &  \multicolumn{1}{c|}{\text{\sffamily X}} & \multicolumn{1}{c|}{\ding{52}} & \multicolumn{1}{c|}{\text{\sffamily X}} &  \multicolumn{1}{c|}{\text{\sffamily X}} & \multicolumn{1}{c|}{\ding{52}} \\ \hline
\textbf{INTERSECT}  &  \multicolumn{1}{c|}{\ding{52}} & \multicolumn{1}{c|}{\ding{52}}  & \multicolumn{1}{c|}{\ding{52}}  &   \multicolumn{1}{c|}{\text{\sffamily X}} & \multicolumn{1}{c|}{\ding{52}} \\ \hline
\textbf{AS} in FROM Clause & \multicolumn{1}{c|}{\ding{52}} &\multicolumn{1}{c|}{\ding{52}} & \multicolumn{1}{c|}{\ding{52}} & \multicolumn{1}{c|}{\text{\sffamily X}} & \multicolumn{1}{c|}{\ding{52}} \\ \hline
\textbf{EXCEPT ALL}  & \multicolumn{1}{c|}{\text{\sffamily X}} &\multicolumn{1}{c|}{\ding{52}} & \multicolumn{1}{c|}{\text{\sffamily X}} &{\begin{tabular}[c]{@{}c@{}}x\\ (MINUS ALL\\  is not supported)\end{tabular}} 
       & \multicolumn{1}{c|}{\ding{52}} \\ \hline
\textbf{EXCEPT} & \multicolumn{1}{c|}{\text{\sffamily X}} & \multicolumn{1}{c|}{\ding{52}} & \multicolumn{1}{c|}{\ding{52}} &  \multicolumn{1}{c|}{\text{\sffamily X}} & \multicolumn{1}{c|}{\ding{52}}\\ \hline

GROUP BY contains columns not in SELECT list &\multicolumn{1}{c|}{\ding{52}} &  \multicolumn{1}{c|}{\text{\sffamily X}} &  \multicolumn{1}{c|}{\text{\sffamily X}} &  \multicolumn{1}{c|}{\text{\sffamily X}} &  \multicolumn{1}{c|}{\text{\sffamily X}} \\ \hline

Arithmetic operations in WHERE Clause &\multicolumn{1}{c|}{\ding{52}} &  \multicolumn{1}{c|}{\text{\sffamily X}} &  \multicolumn{1}{c|}{\text{\sffamily X}}  &  \multicolumn{1}{c|}{\text{\sffamily X}} & \multicolumn{1}{c|}{\ding{52}} \\ \hline

Support keyword True in WHERE clause & \multicolumn{1}{c|}{\ding{52}} & \multicolumn{1}{c|}{\ding{52}} &  \multicolumn{1}{c|}{\text{\sffamily X}} &  \multicolumn{1}{c|}{\text{\sffamily X}} & \multicolumn{1}{c|}{\ding{52}} \\ \hline

Support of \% operator  & \multicolumn{1}{c|}{\ding{52}} & \multicolumn{1}{c|}{\ding{52}} & \multicolumn{1}{c|}{\ding{52}} & \multicolumn{1}{c|}{\begin{tabular}[c]{@{}c@{}}\ding{52}\\ (It uses mod function)\end{tabular}} & \multicolumn{1}{c|}{\ding{52}}\\ \hline

Division by zero & \multicolumn{1}{c|}{\ding{52}} &  \multicolumn{1}{c|}{\text{\sffamily X}} &            \multicolumn{1}{c|}{\text{\sffamily X}} &  \multicolumn{1}{c|}{\text{\sffamily X}} & \multicolumn{1}{c|}{\text{\sffamily X}} \\ \hline

Row comparison & \multicolumn{1}{c|}{\ding{52}} & \multicolumn{1}{c|}{\ding{52}} &  \multicolumn{1}{c|}{\text{\sffamily X}} & \multicolumn{1}{c|}{\ding{52}} & \multicolumn{1}{c|}{\ding{52}} \\ \hline

\textbf{Identical names in the FROM Clause} &  \multicolumn{1}{c|}{\text{\sffamily X}} &  \multicolumn{1}{c|}{\text{\sffamily X}} &  \multicolumn{1}{c|}{\text{\sffamily X}} &  \multicolumn{1}{c|}{\text{\sffamily X}} & \multicolumn{1}{c|}{\ding{52}} \\ \hline

SQL Query without FROM Clause &  \multicolumn{1}{c|}{\text{\sffamily X}}  & \multicolumn{1}{c|}{\ding{52}} & \multicolumn{1}{c|}{\ding{52}} &  \multicolumn{1}{c|}{\text{\sffamily X}} &  \multicolumn{1}{c|}{\text{\sffamily X}} \\ \hline 

\end{tabular}%
}
\end{table}


\begin{table}[h]
\caption{Summarize results}
\label{my-label}
\resizebox{\columnwidth}{!}{%
\begin{tabular}{|l|l|l|l|l|l|}  
\hline
\textbf{Operation} & \textbf{Mysql} & \textbf{PostgreSQL} & \textbf{Microsoft SQL Server} & \textbf{Oracle} & \textbf{IBM DB2} \\ \hline

$||$ operator in SELECT Clause  & \multicolumn{1}{c|}{\begin{tabular}[c]{@{}c@{}}\ding{52}\\ (But  It uses $||$ as logical OR), \\(Uses CONCAT function ) \end{tabular}} & \multicolumn{1}{c|}{\ding{52}} & \multicolumn{1}{c|}{\begin{tabular}[c]{@{}c@{}}\ding{52}\\ ( It uses $+$ )  \end{tabular}} & \multicolumn{1}{c|}{\ding{52}} & \multicolumn{1}{c|}{\ding{52}} \\ \hline


TRIM function in SELECT Clause  & \multicolumn{1}{c|}{\ding{52}} & \multicolumn{1}{c|}{\ding{52}} &  \multicolumn{1}{c|}{\text{\sffamily X}}  & \multicolumn{1}{c|}{\ding{52}}      & \multicolumn{1}{c|}{\ding{52}} \\ \hline


  operator in SELECT Clause  &  
\multicolumn{1}{c|}{\begin{tabular}[c]{@{}c@{}}\ding{52}\\ (But it uses   \\ as logical OR) \end{tabular}} & \multicolumn{1}{c|}{\ding{52}} & \multicolumn{1}{c|}{\ding{52}}  & \multicolumn{1}{c|}{\ding{52}} & \multicolumn{1}{c|}{\ding{52}}  \\ \hline

SUBSTRING function in SELECT Clause & \multicolumn{1}{c|}{\ding{52}} & \multicolumn{1}{c|}{\ding{52}} & \multicolumn{1}{c|}{\ding{52}} & \multicolumn{1}{c|}{\begin{tabular}[c]{@{}c@{}}{\text{\sffamily X}} (It uses SUBSTR \\function instead)\end{tabular}} &     \multicolumn{1}{c|}{\ding{52}}  \\ \hline

Enclose Strings with “ instead of ‘  & \multicolumn{1}{c|}{\ding{52}} &  \multicolumn{1}{c|}{\text{\sffamily X}} &  \multicolumn{1}{c|}{\text{\sffamily X}} &  \multicolumn{1}{c|}{\text{\sffamily X}}  &  \multicolumn{1}{c|}{\text{\sffamily X}}  \\ \hline

LIKE Operator  & \multicolumn{1}{c|}{\ding{52}} & \multicolumn{1}{c|}{\begin{tabular}[c]{@{}c@{}}\ding{52}\\ (Case- sensitive) \end{tabular}} & \multicolumn{1}{c|}{\ding{52}} & \multicolumn{1}{c|}{\begin{tabular}[c]{@{}c@{}}\ding{52}\\ (Case- sensitive) \end{tabular}}      & \multicolumn{1}{c|}{\ding{52}} \\ \hline

\end{tabular}%
}
\end{table}

\hfill\newpage
\section{Suggestions for future work}
Several issues arose when DBMSs are tested. The experiments are conducted in various databases that contained integers and strings. We expect that more issues can arise by generating also databases containing dates but by doing so, the generator tool should be extended in order to support this new data type. This should be an easy extension as there is the provision for supporting any data type. Yet another future extension could be to include a new DBMS for evaluation of its SQL-compliance. This extension also should not need a lot of effort as the architecture in implemented in such a way that a new system can be easily added.


% \include{chap2}
%% ... etc ...

%%%%%%%%
%% Any appendices should go here. The appendix files should look just like the
%% chapter files.
\appendix
\include{appendix1}
%% ... etc...

%% Choose your favourite bibliography style here.
\bibliographystyle{apalike}

%% If you want the bibliography single-spaced (which is allowed), uncomment
%% the next line.
% \singlespace

\bibliographystyle{abbrv} 

\bibliography{thesis}


%% ... that's all, folks!
\end{document}
