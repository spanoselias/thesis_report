\chapter{Strategy and Definitions}
This chapter will describe the main strategy that will be used in order to test the SQL compliance of current DBMSs. 
 
As it was mention, the key idea of the current project is to examine whether current DBMSs respect the SQL standards and to highlight the differences that might exist among those systems. One of the core component that will be used  for testing the SQL compliance is the Random SQL Generator tool. Thus, the first goal is to design and implement such a tool which will generate a diversity of SQL queries by exposing the expressivity of SQL language. It is worthy mentioning that this tool can be important of its own use by utilize it for testing a new DBMSs. The implementation of the tool will be in such a way that a new DBMS can be added without affect almost anything of the current implementation. A detailed explanation about the implementation and the internal structure of the tool is given in the following Chapter. Additionally, by having such a tool, it will be possible to generate thousands of different SQL queries and evaluate them in five systems, namely PostgreSQL, Microsoft SQL Server, IBM DB2, Oracle Database and MySQL for identifying differences. 
Subsequently, another useful tool that it will be designed and implemented is the run and compare tool. Apparently this tool must be compatible with all the DBMSs and it should evaluate each query and compare the results of each DBMS whether they are identical with concerning the rest systems. The comparison of DBMS’s results will be performed using main-memory data structure for achieving efficiency. In case where a difference between any system will be found, then a log file will be created to record all the differences. Again a detailed explanation of tool’s internal implementation is given in the next chapter.
Lastly but not least, a tool is needed for generating random data in order to perform our experiments. For that purpose, datafiller will be used which is a well-known open-source tool that provides the capability of generating realistic data. The data are generated based on a database schema which is provided to the tools as a parameter and many parameters can be specified for generating realistic data with different sizes. 



 