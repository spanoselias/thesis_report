\chapter{Methodology}

\section{Methodology}
 
As it was mentioned in the first chapter, the key idea of the current project is to investigate whether current DBMSs comply according to the Standard and highlight the main issues and differences among popular systems. In particular, five popular systems such as MySQL, IBM DB2, Microsoft SQL Server, PostgreSQL and Oracle Database are examined. Hence, this project providing a complete framework that can be used to systematically test the SQL-Compliance as these systems have been evolved rapidly.
 
The framework consists of three different tools, and more precisely, two different tools have been implemented (1) the Random SQL Query Generator Tool, and (2) the Comparison Tool. The third one, which is used to produce random realistic data is an open-source tool called Datafiller. Different data can be generated using Datafiller tool based on a schema which is provided as a parameter. In addition, complex data can be generated by specifying additional parameters to the tool.  
 
The purpose of the random query generator tool is to rapidly generate a very large number of SQL queries based on the SQL statements as defined in the SQL Standard. In that way, an examination of whether answers to SQL queries are the same independently of which DBMS they are executed on or not can be performed. Therefore, the comparison tool is used to identify if the results are identical among DBMSs. A detailed explanation of the internal implementation and structure of those tools is given in the following Chapter. The implementation of the tool is modular and allows the addition of new DBMSs seamlessly without altering the existing structure of the tool. The comparison of the DBMSs’ results will be performed using a main-memory data structure for achieving efficiency. Differences among current DBMSs are automatically documented by the tool. For example, a query may not be executed on one of the DBMSs and raise an error. Hence, this error is logged into the log file that the comparison tool generates.  








 