\chapter{Methodology}

\section{Methodology}
 
This chapter describes the primary Methodology that is used to test the SQL-Complia- nce of current DBMSs. 

As it was mention in the first chapter, the key idea of the current research is to examine whether current DBMSs comply according to SQL standards and highlight the main differences among those systems. More precisely, it will be examined five popular systems such as MySQL, IBM DB2, Microsoft SQL Server, PostgreSQL and Oracle Database. Aside from that this research has as target  to provide a complete architecture that can be used to systematically inspect the SQL-Compliance as these systems have been evolved rapidly. The architecture is consisted by three different tools. More precisely, it has been implemented two different tools such as Random SQL Query Generator Tool and the Comparison Tool. The third one which is used to produce random realistic data realistic data is open-source tool namely Datafiller. The data can be generated based on a database schema which is given to the tools as a parameter. Complex data can be generated by specifying some parameter to the tool.  The purpose of the SQL generator tool is to generate a diversity of SQL queries by using SQL commands which are defined in the SQL standard. In that way, it can be examined whether answers to SQL queries are the same independently of which DBMS they are executed on. Therefore, the comparison tool is used to identify if the results are identical. A detailed explanation of the internal implementation and structure of those tools is given in the following Chapter. The implementation of the tool is in such a way that a new DBMS or functionality could be added without affecting the rest implementation.The comparison of DBMS’s results will be performed using a main-memory data structure for achieving efficiency. Differences among current DBMSs are automatically documented by the tool. For example, a query may not be executed on one of the DBMSs and raise an error. Hence, this error is logged into the log file that the comparison tool generates. 








 