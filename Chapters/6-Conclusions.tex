\chapter{Conclusions}

\section{Conclusions}

In this project an entire architecture is implemented and is used to evaluate the SQL-compliance for five DBMSs. Also, we verified the correctness and efficiency of implemented tools by conducting the experiments and we demonstrated from the experiment evaluation chapter that the implemented tools are competent to reveal crucial differences among current systems. It worthy mentioning that without a similar architecture, it would be almost impossible to detect some of the differences and issues. Further, as described in the related work, there is no a similar architecture except of some documentations provided by the vendors of such systems, and some other studies which presented some issues according to the Standard without having a systematic tool. In addition, we provide shortly a summarize table representing all the issues and incompatibilities between the most popular DBMSs which are of major importance for vendors, users and programmers of such systems.  

The goal of this project is achieved and more than twenty issues and incompatibilities have been disclosed which make clear that some parts of the Standard are implemented differently. In addition, the implemented tools can be major importance for future vendors or researchers.

Furthermore, demonstrating and analyzing these incompatibilities make aware both users and programmers for these issues. We verified our assumptions that some parts of the Standard are implemented differently but it was somewhat surprisingly that so many differences have emerged which makes clear that the standard is difficult to be interpreted and implemented in exactly the same way by all DBMSs. In addition, awareness of these differences issues may cause irreversible consequences in companies if we take into account that such systems are utilized in almost every field.  

Aside from the issues which have been detected, the implemented architecture is portable and they can extended efficiently. For example, although it is provided experimental evidences for numbers and strings data types, the random generator tool is implemented in such a way that can track any data type such as Date. In that way, it can be extended efficiently to generate queries with attributes of date as data type. 


\section{Summary of the findings}


\section{Suggestions for future work}
Several issues arose when DBMSs are tested. The experiments are conducted in various databases that contained integers and strings. We expect that more issues can arise by generating also databases containing dates but by doing so, the generator tool should be extended in order to support this new data type. This should be an easy extension as there is the provision for supporting any data type. Yet another future extension could be to include a new DBMS for evaluation of its SQL-compliance. This extension also should not need a lot of effort as the architecture in implemented in such a way that a new system can be easily added.

